\documentclass[13pt,hyperref={unicode,naturalnames}]{beamer}
\usepackage{fsr}

\makeatletter
\def\fsr@institute{Факультет космических исследований\\
  Московский Государственный Университет им. М.В. Ломоносова}
\def\fsr@instituteshort{ФКИ}
\def\fsr@subject{Предмет}
\def\fsr@semestr{0}
\def\fsr@lector{Иванов И.\,И.}
\def\fsr@texedby{А.\,А.~Петров\inst{1}, Б.\,Б.~Сидоров\inst{1}}
\def\fsr@texedbyshort{Петров, Сидоров}
\def\fsr@title{Заголовок}
\def\fsr@subtitle{Подзаголовок}
\def\fsr@date{Предмет, январь 2000\,г.}
\def\fsr@shortdate{2000}
% \def\fsr@enabletransfer{\true} % разрешить автоперенос без дублирования после всех знаков операций в формулах
\def\fsr@standartmargins{\true} % стандартный размер полей
\def\fsr@presentation{\true}
\def\fsr@numeration{\true} % включить нумерацию страниц (или шапку презентации)
% \def\fsr@standartfont{\true} % вернуть стандартный шрифт вместо виноградовского
\fsrinitialize
\makeatother

\begin{document}
\frame{\titlepage}
%---------------------------------------------------------
\begin{frame}
\frametitle{Содержание}
\tableofcontents
\end{frame}
%---------------------------------------------------------
\section{Раздел 1}
%---------------------------------------------------------
\begin{frame}{Слайд 1}
    Текст до паузы\\
    \pause
    Текст после паузы
\end{frame}
%---------------------------------------------------------
\section{Раздел 2}
%---------------------------------------------------------
\begin{frame}
\frametitle{Слайд 1}
Важный текст можно
\alert{выделить}.
\begin{block}{Ремарка}
Текст
\end{block}
\begin{alertblock}{Важная штука}
Обычно красная, но не здесь
\end{alertblock}
\begin{exampleblock}{Примеры}
Ееее
\end{exampleblock}
\end{frame}
%---------------------------------------------------------
\begin{frame}
\frametitle{Слайд на 2 колонки}
\begin{columns}
\column{0.5\textwidth}
Текст в 1 колонке.
$$E=mc^2$$
\column{0.5\textwidth}
Текст во 2 колонке.
\end{columns}
\end{frame}
\end{document}